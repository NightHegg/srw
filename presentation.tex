\documentclass{beamer}
\usepackage[english, russian]{babel}
\usepackage[utf8]{inputenc}
\usepackage[english,russian]{babel}
\setbeamertemplate{caption}[numbered]
\usepackage{comment}
\usepackage{amsmath,amsfonts,amssymb,mathtext,cite,enumerate,float,indentfirst,floatflt,setspace}
\usepackage{geometry}
\usepackage{multicol,subcaption,multirow}
\usepackage{graphicx,tocvsec2}
\usepackage[skip=2pt,font=tiny]{caption}
\usepackage{adjustbox,wrapfig,bm}
\usepackage{enumitem,layout}

\setlist{nolistsep}
\usefonttheme[onlymath]{serif}
\hypersetup{unicode=true}
\usetheme{Boadilla}
\usecolortheme{rose}
\makeatletter
\newcommand\titlegraphicii[1]{\def\inserttitlegraphicii{#1}}
\titlegraphicii{}
\setbeamertemplate{title page}
{
  \vbox{}
   {\usebeamercolor[fg]{titlegraphic}\inserttitlegraphic\hfill\inserttitlegraphicii\par}
  \begin{centering}
    \begin{beamercolorbox}[sep=8pt,center]{institute}
      \usebeamerfont{institute}\insertinstitute
    \end{beamercolorbox}
    \begin{beamercolorbox}[sep=8pt,center]{title}
      \usebeamerfont{title}\inserttitle\par%
      \ifx\insertsubtitle\@empty%
      \else%
        \vskip0.25em%
        {\usebeamerfont{subtitle}\usebeamercolor[fg]{subtitle}\insertsubtitle\par}%
      \fi%     
    \end{beamercolorbox}%
    \vskip1em\par
    \begin{beamercolorbox}[sep=8pt,center]{author}
      \usebeamerfont{author}\insertauthor
    \end{beamercolorbox}
  \end{centering}
  %\vfill
}
\makeatother

\setbeamertemplate{footline}
{
  \leavevmode%
  \hbox{%
	\begin{beamercolorbox}[wd=.9\paperwidth,ht=2.25ex,dp=1ex,center]{author in head/foot}%
    \usebeamerfont{scriptsize}Применение различных вариантов метода декомпозиции для численного решения задач деформирования упругих тел
  \end{beamercolorbox}%
  \begin{beamercolorbox}[wd=.1\paperwidth,ht=2.25ex,dp=1ex,right]{title in head/foot}%
    \insertframenumber{} / \inserttotalframenumber\hspace*{1ex}
  \end{beamercolorbox}}%
  \vskip0pt%
}
	
\setbeamertemplate{navigation symbols}{}

\title{Применение различных вариантов  \\ метода декомпозиции \\ для численного решения \\ задач деформирования упругих тел}
\institute{МИНИСТЕРСТВО ОБРАЗОВАНИЯ И НАУКИ \\ РОССИЙСКОЙ ФЕДЕРАЦИИ \\ Федеральное агентство по образованию \\  МОСКОВСКИЙ ГОСУДАРСТВЕННЫЙ ТЕХНИЧЕСКИЙ УНИВЕРСИТЕТ ИМЕНИ Н.Э. БАУМАНА \\ Факультет "`Фундаментальные науки"' \\ Кафедра "`Прикладная математика"'}
\date{Москва, 2018 г.}

\begin{document}

\begin{frame}[plain]
\maketitle
\tiny
\begin{tabular}[t]{@{\hspace{150pt}}l@{\hspace{10pt}}l@{}}
Исполнитель: & Матвеев Михаил \\
Научный руководитель: & канд. физ.-мат. наук  \\
& Родин Александр Сергеевич
\end{tabular}
\centering
\bigskip 

\insertdate
\end{frame}
\begin{frame}
\small
\begin{block}{Цель}
Сравнение численных решений задачи нагружения трубы давлением, полученных с использованием стандартного и смешанного методов конечных элементов, в упругой постановке и в постановке с учётом деформации ползучести.
\end{block}
\begin{block}{Задачи}
\begin{itemize}
\item[-]изучение модели упругого материала и модели материала, учитывающую деформации ползучести;
\medskip
\item[-]анализ применения стандартного и смешанного МКЭ к задаче нагружения упругой трубы давлением;
\medskip
\item[-]анализ применения стандартного и смешанного МКЭ к задаче нагружения давлением трубы с учётом деформации ползучести;
\end{itemize}
\end{block}
\end{frame}

\begin{frame}{Постановка задачи механики твёрдого деформируемого тела}
\begin{multicols}{2}
\small
Уравнения равновесия в общем виде:
\begin{equation*}
\frac{\partial \sigma_{ji}}{\partial x_{j}}=0
\end{equation*}
с кинематическим условием ($S_{1}$) и с силовым условием ($S_{2}$)
\begin{equation*}
u=u_{0}, \qquad \sigma_{ji} n_{j}=p_{i},
\end{equation*}
где $\sigma_{ji}$-компоненты тензора напряжений, $p_i$-компоненты вектора поверхностных сил, $x_{j}$-декартовы координаты. 

Уравнения равновесия в цилиндрической системе координат:
\begin{equation*}
\frac{d}{d\:r}(\sigma_{rr}\: r)-\sigma_{\varphi\varphi}=0.
\end{equation*}
\columnbreak

\begin{figure}[h]
\centering
\includegraphics[width=0.2\textwidth]{PostZ.png}
\captionsetup{font=tiny}
\caption{Труба под внутренним и внешним давлением}
\end{figure}
 \vspace{-1em}
Считаем, что все переменные не зависят от координат $\varphi$ и $z$, а также осевая деформация трубы равна нулю: $\varepsilon_{zz}=0$.
\end{multicols}
\end{frame}

\begin{frame}{Математическая модель упругого тела}
\small
Упругие деформации совпадают с полными. Закон Гука:

\begin{equation*}
\sigma_{ij}=C_{ijkl}\varepsilon_{kl}.
\end{equation*}
Для изотропного тела с учётом принятых допущений

\begin{equation*}
\begin{split}
\varepsilon_{rr}=\frac{du}{dr}, \qquad \sigma_{rr}=(\lambda+2\mu)\varepsilon_{rr}+\lambda\varepsilon_{\varphi\varphi}, \\
\varepsilon_{\varphi\varphi}=\frac{u}{r}, \qquad \sigma_{\varphi\varphi}=(\lambda+2\mu)\varepsilon_{\varphi\varphi}+\lambda\varepsilon_{rr}, \\ 
\varepsilon_{zz}=0, \qquad \sigma_{zz}=\left(\lambda+\mu\right)(\varepsilon_{rr}+\varepsilon_{\varphi\varphi}),
\end{split}
\end{equation*}

где $\lambda$, $\mu$-параметры Ламе.

\end{frame}

\begin{frame}{Математическая модель материала с учётом деформации ползучести}
\small
Аддитивное разложение тензора полной деформации:
\begin{equation*}
\varepsilon_{ij}=\varepsilon_{ij}^{e}+\varepsilon_{ij}^{c},
\end{equation*}
где $\varepsilon_{ij}^{c}$-компоненты тензора деформации ползучести. Закон Гука:
\begin{equation*}
\sigma_{ij}=C_{ijkl}(\varepsilon_{kl}-\varepsilon^{c}_{kl}),
\end{equation*}

Закон ползучести:
\begin{equation*}
\dot{\varepsilon}_{ij}^{c}=\tilde{\lambda}\acute{\sigma_{ij}}.
\end{equation*}
Параметр ползучести: 
\begin{equation*}
\tilde{\lambda}=\dfrac{3}{2}B\sigma_{u}^2,
\end{equation*}
где $B$-функция времени, которая определяется выбором материала. 

Деформация ползучести несжимаемая: $\varepsilon^{c}_{ii}=0$.
\end{frame}

\begin{frame}{Численное решение задачи упругости стандартным МКЭ}
\small
Дифференциальное уравнение равновесия решим с помощью метода Бубнова-Галёркина, получив в итоге уравнение в матричном виде
\begin{equation*}
\mathbf{Ku}=\mathbf{f},
\end{equation*}
где 
\vspace{-0.2em}
\begin{table}[h]
\begin{tabular}{p{9em}l}
матрица жёсткости: & $\mathbf{K}$=$\mathbf{B}^T$ $\mathbf{D}\mathbf{B}$ \\ 
вектор деформаций: & $\bm{\varepsilon}$=
$
\begin{pmatrix}
\varepsilon_{r r} \\
\varepsilon_{\varphi \varphi}
\end{pmatrix}
$
=$\mathbf{B}\mathbf{u}$ \\
\multicolumn{2}{c}{
$ \hspace{3em} \mathbf{B}=\mathbf{S}\mathbf{N}$=
$
\begin{pmatrix}
\partial / \partial r & \partial / \partial r \\
1/r & 1/r
\end{pmatrix}
\begin{pmatrix}
N_1 \\
N_2
\end{pmatrix}
$
} \\
вектор напряжений: & $\bm{\sigma}$=
$
\begin{pmatrix}
\sigma_{rr} \\
\sigma_{\varphi \varphi} \\
\sigma_{zz}
\end{pmatrix}
$
=$\mathbf{D}\bm{\varepsilon}$ \\
матрица коэфф-ов упругости: & $\mathbf{D}=\dfrac{E}{1-\nu^2}$
$
\begin{pmatrix}
1 & \nu \\
\nu & 1  \\
\nu & \nu 
\end{pmatrix}
$
\end{tabular}
\end{table}

\end{frame}

\begin{frame}{Численное решение задачи упругости смешанным МКЭ}
\small
Рассмотрим давление как независимую переменную. Для изотропного случая объёмная деформация 
\begin{equation*}
\varepsilon_v=\dfrac{\varepsilon_{ii}}{3}=\dfrac{p}{K}, 
\end{equation*}
где $K$-модуль объёмной упругости. 

Уравнение равновесия в матричной форме
\begin{equation*}
\begin{bmatrix}
\mathbf{A} & \mathbf{C} \\
\mathbf{C}^{T} & -\mathbf{V} 
\end{bmatrix}
\begin{Bmatrix}
\tilde{\mathbf{u}} \\
\tilde{\mathbf{p}} 
\end{Bmatrix}
=
\begin{Bmatrix}
\mathbf{f}_1 \\
0
\end{Bmatrix},
\end{equation*}
где

\begin{equation*}
\begin{split}
\mathbf{A}=\int_{\Omega} {\mathbf{B}^{T} \mathbf{D}_{d} \mathbf{B} d\Omega}, \qquad \mathbf{C}=\int_{\Omega} {\mathbf{B}^{T} \mathbf{m} \mathbf{N_{p}} d\Omega},\\
\mathbf{V}=\int_{\Omega} {\mathbf{N_{p}^{T}} \dfrac{1}{K} \mathbf{N_{p}} d\Omega}, \qquad \mathbf{f}_1=\int_{\Gamma} {\mathbf{N_{u}^{T}} \bar{\mathbf{t}}d\Gamma},
\end{split}
\end{equation*}
$N_u$-линейная базисная функция для аппроксимации перемещения, $N_p$-кусочно-постоянная базисная функция для аппроксимации давления.
\end{frame}

\begin{frame}{Численное решение краевой задачи с учётом деформации ползучести}
\small
Учёт деформации ползучести приводит к нелинейной задаче. Линеаризация-метод начальной деформации. Уравнения равновесия в каждый момент времени $t_{k+1}$ в матричном виде для стандартного МКЭ
\begin{equation*}
\textbf{Ku}=\textbf{f}+\textbf{f}^{c}
\end{equation*}
и для смешанного МКЭ
\begin{equation*}	
\begin{bmatrix}
\mathbf{A} & \mathbf{C} \\
\mathbf{C}^{T} & -\mathbf{V} 
\end{bmatrix}
\begin{Bmatrix}
\tilde{\mathbf{u}} \\
\tilde{\mathbf{p}} 
\end{Bmatrix}
=
\begin{Bmatrix}
\mathbf{f}_1 \\
0
\end{Bmatrix}
+
\begin{Bmatrix}
\mathbf{f}_1^{c} \\
0
\end{Bmatrix},
\end{equation*}
где 
\begin{equation*}
\mathbf{f}^{c}=\int_{\Omega} {\mathbf{B}^{T} \mathbf{D} \bm{\varepsilon}^{c}(t_{k}) d\Omega},
\end{equation*}
\begin{equation*}
\mathbf{f}_1^{c}=\int_{\Omega} {\mathbf{B}^{T} \mathbf{D}_{d} \bm{\varepsilon}^{c}(t_{k}) d\Omega}.
\end{equation*}
\end{frame}

\begin{frame}{Результаты решения для упругой трубы}
\small
\framesubtitle{Основные данные}
Внутреннее давление $p_a=20$ МПа, внешнее давление $p_b=0$ МПа. Не учитывается осевая растягивающая сила: $\sigma_{zz}=0$.

Аналитические решения:
\begin{equation*}
u=\frac{\left(1-2\nu\right)\left(1+\nu\right)}{E} \frac{p_a a^2}{b^2-a^2}r+\frac{1+\nu}{E}\frac{a^2 b^2}{r}\frac{p_a}{b^2-a^2},
\end{equation*}
\begin{equation*}
\sigma_{rr}=\frac{p_a a^2}{b^2-a^2}-\frac{a^2 b^2}{r^2}\frac{p_a}{b^2 -a^2},
\end{equation*}
\begin{equation*}
\sigma_{\varphi\varphi}=\frac{p_a a^2}{b^2-a^2}+\frac{a^2 b^2}{r^2}\frac{p_a}{b^2 -a^2}.
\end{equation*}	

Численные расчёты проведены:
\smallskip
\begin{itemize}
\item[-]на трёх сетках (h=0,02; h=0,01; h=0,005);
\smallskip
\item[-]двумя методами (стандартный и смешанный МКЭ);
\smallskip
\item[-]для разных значений коэффициента Пуассона \\ ($\nu$=0,34; $\nu$=0,4999; $\nu$=0,49999; $\nu$=0,5);
\end{itemize}
\end{frame}

\begin{frame}{Результаты решения для упругой трубы}
\framesubtitle{Коэффициент Пуассона $\nu$=0.34}

\begin{multicols}{2}
\begin{figure}[h]
\centering
\includegraphics[width=0.43\textwidth]{ELR1_1.png}
\caption{Зависимость радиальных напряжений от радиуса}
\end{figure}
\vspace{-2em}
\begin{figure}
\centering
\includegraphics[width=0.43\textwidth]{ELT1_1.png}
\caption{Зависимость окружных напряжений от радиуса}
\end{figure}

\columnbreak

\begin{table}[h]	
\begin{center}
\begin{adjustbox}{max width=0.4\textwidth}
\begin{tabular}{|@{}c@{}|@{\hspace{0.1em}}c@{}|@{\hspace{0.3em}}c@{\hspace{0.3em}}|@{\hspace{0.3em}}c@{\hspace{0.3em}}|}
\hline	
МКЭ &h, мм & $\sigma_{r}$, МПа &  $\sigma_{\varphi}$, МПа \\ \hline
\multirow{3}{*}{Стан.}
& 0.02 & 2.74$\times 10^{-2}$& 1.07$\times 10^{-2}$ \\ \cline{2-4}
& 0.01 & 1.38$\times 10^{-2}$& 5.37$\times 10^{-3}$ \\ \cline{2-4}
& 0.005& 6.9$\times 10^{-3}$ & 2.68$\times 10^{-3}$ \\ \hline
\multicolumn{4}{|c|}{}\\[-0.5em]
\hline
\multirow{3}{*}{Смеш.}
&0.02 & 2.72$\times 10^{-2}$ & 1.07$\times 10^{-2}$ \\ \cline{2-4}
&0.01 & 1.37$\times 10^{-2}$ & 5.36$\times 10^{-3}$ \\ \cline{2-4}
&0.005&  6.9$\times 10^{-3}$ & 2.68$\times 10^{-3}$ \\ \hline
\end{tabular}
\end{adjustbox}
\caption{Норма отн. ошибки в $L_2$}
\end{center}
\end{table}
\vspace*{-5mm}
\small
{
$\varepsilon^{*}_{V}=\dfrac{\varepsilon_{V}}{\varepsilon_{u}}$-оценка отношения объёмной и девиаторной частей деформации; \\ 
%($\varepsilon_{V}=\dfrac{\varepsilon_{ii}}{3}$, $\varepsilon_{u}=\sqrt{\dfrac{2}{3}\varepsilon_{ij}\varepsilon_{ij}}$); 
}
\vspace*{-3mm}
\small{
\begin{table}
\begin{tabular}{r@{}l}
min$\left|\varepsilon^{*}_{V}\right|$= & 0,046 \\
max$\left|\varepsilon^{*}_{V}\right|$= & 0,183 \\
\end{tabular}
\end{table}
}

\end{multicols}

\end{frame}

\begin{frame}{Результаты решения для упругой трубы}
\framesubtitle{Коэффициент Пуассона $\nu$=0.49999}

\begin{multicols}{2}
\begin{figure}[h]
\centering
\includegraphics[width=0.44\textwidth]{ELR3_1.png}
\caption{Зависимость радиальных напряжений от радиуса}
\end{figure}
\vspace{-2em}
\begin{figure}
\centering
\includegraphics[width=0.44\textwidth]{ELT3_1.png}
\caption{Зависимость окружных напряжений от радиуса}
\end{figure}

\columnbreak

\begin{table}[h]	
\begin{center}
\begin{adjustbox}{max width=0.4\textwidth}
\begin{tabular}{|@{}c@{}|@{\hspace{0.1em}}c@{}|@{\hspace{0.3em}}c@{\hspace{0.3em}}|@{\hspace{0.3em}}c@{\hspace{0.3em}}|}
\hline
МКЭ &h, мм & $\sigma_{r}$, МПа &  $\sigma_{\varphi}$, МПа \\ \hline
\multirow{3}{*}{Стан.}
& 0.02  & 1.05 & 3.27$\times 10^{-1}$ \\ \cline{2-4}
& 0.01  & 4.33$\times 10^{-1}$ & 1.34$\times 10^{-1}$ \\ \cline{2-4}
& 0.005 & 1.32$\times 10^{-1}$ & 4.07$\times 10^{-2}$ \\ \hline
\multicolumn{4}{|c|}{}\\
\hline
\multirow{3}{*}{Смеш.}
&0.02 & 2.72$\times 10^{-2}$ & 1.07$\times 10^{-2}$ \\ \cline{2-4}
&0.01 & 1.37$\times 10^{-2}$ & 5.36$\times 10^{-3}$ \\ \cline{2-4}
&0.005& 6.9$\times 10^{-3}$ & 2.68$\times 10^{-3}$ \\ \hline
\end{tabular}
\end{adjustbox}
\caption{Норма отн. ошибки в $L_2$}
\end{center}
\end{table}

\vspace{-1em}
\small
{
\begin{table}
\begin{tabular}{r@{}l}
min$\left|\varepsilon^{*}_{V}\right|$= & 5.45$\times 10^{-6}$ \\
max$\left|\varepsilon^{*}_{V}\right|$= & 1.27$\times 10^{-5}$ \\
\end{tabular}
\end{table}
}
\end{multicols}
\end{frame}

\begin{frame}{Результаты решения для упругой трубы}
\framesubtitle{Анализ результатов}
\small
\begin{block}{Смешанный МКЭ}
\begin{itemize}
\item[-]при любом коэффициенте Пуассона ошибки не меняются (на одинаковых сетках);
\smallskip
\item[-]для перемещений наблюдается квадратичная скорость сходимости, для напряжений-линейная скорость сходимости.
\end{itemize}
\end{block}
\medskip

\begin{block}{Стандартный МКЭ}
\begin{itemize}
\item[-]при коэффициенте $\nu=0,34$ ошибки такие же, как в смешанном МКЭ;
\smallskip
\item[-]чем ближе коэффициент Пуассона $\nu$ к 0,5, тем больше ошибки (на одинаковых сетках);
\smallskip
\item[-]при измельчении сетки ошибки уменьшаются, но скорости сходимости другие.
\end{itemize}
\end{block}

\end{frame}

\begin{frame}{Результаты решения задачи для трубы с учётом деформации ползучести}
\framesubtitle{Основные данные}
\small
Для случая установившейся ползучести пренебрегаем упругими деформациями. Материал несжимаемый. В законе ползучести n=3, тогда аналитические формулы:
\begin{equation*}
\sigma_{rr}=\frac{p_a\: a^{\tfrac{2}{3}}}{b^{\tfrac{2}{3}}-a^{\tfrac{2}{3}}}\left(1-\frac{b^{\tfrac{2}{3}}}{r^{\tfrac{2}{3}}}\right),
\end{equation*}
\begin{equation*}
\sigma_{\varphi\varphi}=\dfrac{p_a\: a^{\tfrac{2}{3}}}{b^{\tfrac{2}{3}}-a^{\tfrac{2}{3}}}\left(1-\frac{1}{3}\frac{b^{\tfrac{2}{3}}}{r^{\tfrac{2}{3}}}\right).
\end{equation*}
Численные расчёты проведены:
\smallskip
\begin{itemize}
\item[-]на трёх сетках (h=0,02; h=0,01; h=0,005);
\smallskip
\item[-]двумя методами (стандартный и смешанный МКЭ);
\smallskip
\item[-]для значения коэффициента Пуассона $\nu$=0,34;
\end{itemize}

\end{frame}

\begin{frame}{Результаты решения задачи для трубы с учётом деформации ползучести}
\framesubtitle{Основные данные}
\small

\begin{multicols}{2}
\begin{figure}[h]
\centering
\includegraphics[width=0.42\textwidth]{ErrSigmaR_1.png}
\captionsetup{font=tiny}
\caption{Зависимость нормы относительной ошибки $\sigma_{rr}$ от времени для сетки N=25}
\end{figure}
\vspace{-2em}
\begin{figure}
\centering
\includegraphics[width=0.42\textwidth]{ErrSigmaR_2.png}
\captionsetup{font=tiny}
\caption{Зависимость нормы относительной ошибки $\sigma_{\varphi\varphi}$ от времени для сетки N=50}
\end{figure}

\columnbreak

Происходит перестройка решения от упругого случая к случаю установившейся ползучести.В момент времени T=1500 часов численное и аналитическое решения максимально близки, в дальнейшем аналитическое и численное решения начинают расходиться. 
\medskip

В момент времени 50 тысяч часов $max\left|\varepsilon_{rr}\right|\approx 5\%$, после этого деформации перестают быть малыми.

\end{multicols}

\end{frame}

\begin{frame}{Результаты решения задачи для трубы с учётом деформации ползучести}
\framesubtitle{Момент времени T=1500 часов}

\begin{multicols}{2}
\begin{figure}[h]
\centering
\includegraphics[width=0.38\textwidth]{CSR1_1.png}
\caption{Зависимость радиальных напряжений от радиуса}
\end{figure}
\vspace{-1.9em}
\begin{figure}
\centering
\includegraphics[width=0.38\textwidth]{CST1_1.png}
\caption{Зависимость окружных напряжений от радиуса}
\end{figure}

\columnbreak

\begin{table}[h]	
\begin{center}
\begin{adjustbox}{max width=0.4\textwidth}
\begin{tabular}{|@{}c@{}|@{\hspace{0.1em}}c@{}|@{\hspace{0.3em}}c@{\hspace{0.3em}}|@{\hspace{0.3em}}c@{\hspace{0.3em}}|}
\hline
МКЭ &h, мм & $\sigma_{r}$, МПа &  $\sigma_{\varphi}$, МПа \\ \hline
\multirow{3}{*}{Стан.}
& 0.04  & 0.04614   & 0.006752 \\ \cline{2-4}
& 0.02  & 0.02311  & 0.004228 \\ \cline{2-4}
& 0.01 & 0.01176  & 0.003169 \\ \hline
\multicolumn{4}{|c|}{}\\ 
\hline
\multirow{3}{*}{Смеш.}
&0.04  & 0.04476 & 0.007056 \\ \cline{2-4}
&0.02  & 0.02276 & 0.004283 \\ \cline{2-4}
&0.01 & 0.01168 & 0.003175 \\ \hline
\end{tabular}
\end{adjustbox}
\caption{Норма отн. ошибки в $L_2$}
\end{center}
\end{table}
\vspace{-2em}
\small{
\begin{table}
\begin{tabular}{r@{}l}
min$\left|\varepsilon^{*}_{V}\right|$= & 1.04$\times 10^{-2}$ \\
max$\left|\varepsilon^{*}_{V}\right|$= & 0.28 \\
\end{tabular}
\end{table}
}

\end{multicols}

\end{frame}

\begin{frame}{Результаты решения задачи для трубы с учётом деформации ползучести}
\framesubtitle{Момент времени T=50000 часов}

\begin{multicols}{2}
\begin{figure}[h]
\centering
\includegraphics[width=0.38\textwidth]{CSR3_1.png}
\caption{Зависимость радиальных напряжений от радиуса}
\end{figure}
\vspace{-1.9em}
\begin{figure}
\centering
\includegraphics[width=0.38\textwidth]{CST3_1.png}
\caption{Зависимость окружных напряжений от радиуса}
\end{figure}

\columnbreak

\begin{table}[h]	
\begin{center}
\begin{adjustbox}{max width=0.4\textwidth}
\begin{tabular}{|@{}c@{}|@{\hspace{0.1em}}c@{}|@{\hspace{0.3em}}c@{\hspace{0.3em}}|@{\hspace{0.3em}}c@{\hspace{0.3em}}|}
\hline
МКЭ &h, мм & $\sigma_{rr}$, МПа &  err$^{*}$ \\ \hline
\multirow{3}{*}{Стан.}
& 0.04  & 0.09862 & 113 $\%$  \\ \cline{2-4}
& 0.02  & 0.03653 & 58 $\%$ \\ \cline{2-4}
& 0.01 & 0.01477 & 25 $\%$ \\ \hline
\multicolumn{4}{|c|}{}\\ 
\hline
\multirow{3}{*}{Смеш.}
&0.04  & 0.06174 & 38 $\%$ \\ \cline{2-4}
&0.02  & 0.02674 & 17 $\%$ \\ \cline{2-4}
&0.01 & 0.01229 & 5 $\%$ \\ \hline
\end{tabular}
\end{adjustbox}
\caption{Норма отн. ошибки для $\sigma_{rr}$ в $L_2$}
\end{center}
\end{table}
\vspace{-1.8em}
\small
{
err$^{*}$ - показатель увеличения нормы относительной ошибки для $\sigma_{rr}$ по сравнению с моментом времени T=1500.
\vspace{-3em}
\begin{table}
\begin{tabular}{r@{}l}
min$\left|\varepsilon^{*}_{V}\right|$= & 3.56$\times 10^{-4}$ \\
max$\left|\varepsilon^{*}_{V}\right|$= & 9.98$\times 10^{-3}$ \\
\end{tabular}
\end{table}

}
\end{multicols}

\end{frame}

\begin{frame}{Результаты решения задачи для трубы с учётом деформации ползучести}
\framesubtitle{Анализ результатов}
\small
\begin{itemize}
\item[-]для смешанного и стандартного МКЭ в момент времени T=1500 часов ошибки идентичные (на одинаковых сетках), для напряжений наблюдается линейная скорость сходимости;
\medskip
\item[-]для обоих методов с течением времени численное решение начинает расходиться с аналитическим, причём чем мельче сетка, тем медленнее растёт ошибка;
\medskip
\item[-]для смешанного метода ошибка увеличивается значительно медленнее, чем при использовании стандартного метода.
\end{itemize}
\end{frame}

\begin{frame}{Заключение}
\small
\begin{itemize}
\item[-]рассмотрены уравнения равновесия для модели упругого материала и модели материала, учитывающей деформацию ползучести, в квазиодномерном случае;
\medskip
\item[-]построены численные модели с использованием стандартного и смешанного МКЭ;
\medskip
\item[-]данные модели реализованы в виде программы, написанной на языке C++;
\medskip
\item[-]проведены серии расчётов для задачи нагружения давлением упругой трубы и трубы с учётом деформации ползучести;
\medskip
\item[-]проведены анализ полученных результатов и сравнение с аналитическими решениями.
\end{itemize}
\end{frame}

\begin{frame}{Спасибо за внимание!}

\end{frame}

\begin{frame}{Приложение}
\framesubtitle{Блок-схема программы}
\begin{figure}
\centering
\includegraphics[scale=0.2]{Main_Alg.png}
\end{figure}
\end{frame}

\end{document}
